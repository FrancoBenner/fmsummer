% June 17
\chapter{Systemic Risk 1 - Fouque}
\section{History}
\begin{enumerate}
	\item 60's and 70's: 
	\begin{enumerate}
		\item Problem of Portfolio Allocation (Mean-Variance/Markovitz/)
		\item Option Pricing(1973) (Black-Scholes/etc)
	\end{enumerate}
	\item 90's:
	\begin{enumerate}
		\item Local Volatility
		\item Stochastic Volatility
	\end{enumerate}
	\item 2000's:
	\begin{enumerate}
		\item Credit - Taking into account the possibility of default(of company, counterparty, etc)
		\item Credit Basket - structuring risk(Mortgages)
		\item Credit Default Swap
		\item Collatorized Default Options - Huge Huge Market.
		\item Financial Crisis - Mortgages were completely mispriced.
	\end{enumerate}
	
	\item 2008-2010:
	\begin{enumerate}
		\item NIF: National Institute of Finance (for doing research on the banking \emph{system})
		\item Handbook on Systemic Risk: Cambridge
		\item Dodd-Frank: Created Office of Financial Research (under the treasury department)
	\end{enumerate}
\end{enumerate}

% \begin{definition}\label{def:systemic_risk}
% 	We define \emph{systemic risk}: 
% 	\begin{enumerate}
% 		\item a lot of defaults
% 		\item lack of liquidity
% 	\end{enumerate}
% \end{definition}

\section{Systemic Risk}
Can consider from many views: mathematics, statistics, etc.

Two main approaches:
\begin{enumerate}
	\item Coupled Diffusions: Continuous time
	\item Networks
\end{enumerate}

The starting point is Brownian Motion. Suppose we start with an asset:
\begin{equation}
	dS_t = \mu S_t dt
\end{equation}
which we can solve by
\begin{equation}
	S_t = S_0e^{\mu t}
\end{equation}
Okay, but usually we don't know the $\mu$, and there is some noise! How can we add noise? Since $\mu$ is the return, maybe by adding some \emph{white noise}
\begin{equation}
	dS_t = S_t(\mu dt + \text{Noise})
\end{equation}
where the Noise term is given by Brownian Motion, which has the form
\begin{equation}
	\sigma dW_t
\end{equation}
with $\sigma > 0$, and $(W_t)_{t\geq 0}$ for $t\leq T$. The properties of Brownian Motion:
\begin{enumerate}
	\item $W_0=0$
	\item $W_t$ is continuous
	\item Independent, increments
	\item If we consider
	\begin{equation}
		0 < t_1 < ... < t_n \leq T
	\end{equation}
	then it gives rise to many differences:
	\begin{equation}
		(W_{t_1} - W_{t_0}, W_{t_2} - W_{t_1}, ... W_{t_n} - W_{t_{n-1}})
	\end{equation}
	such that
	\begin{equation}
		\mathcal{D}(w_t - w_s) = \mathcal{N}(0, t-s)
	\end{equation}
\end{enumerate}
A bit of Brownian Motion History:
\begin{enumerate}
	\item Brown
	\item Buchelier (1900)
	\item Albert Einstein (1905): Heat Equation/Brownian Motion tie-in
	\item Weiner (1930s): Constructed Brownian Motion: Construct a measure over all continuous trajectories
	\item Ito (1940s): Figures out chain rule for brownian motion
	\item Samuelson (1960s): Generalized Brownian Motion
\end{enumerate}

If we have a bounded function $f(t)$ where
\begin{equation}
	\sum_{0}^T \left( f(t_{i+1}) - f(t_i) \right) < \infty
\end{equation}
If we have a brownian motion instead,
\begin{equation}
	\sum_0^T \left| W_{t_{i+1}} - W_{t_i} \right| \to \infty
\end{equation}

Book Reference: Carson \& Tree???

\section{Hitting Times}
Draw a plot:
\begin{enumerate}
	\item time on horizontal axis
	\item y = brownian motion on vertical axis
	\item line $y=\alpha$.
\end{enumerate}

\begin{figure}[htbp]
	\centering
		\includegraphics[width=.9\linewidth]{brownian_motion.png}
	\caption{caption}
	\label{fig:label}
\end{figure}

We are interested in the time $\tau_{\alpha}$, defined by
\begin{equation}
	\tau_a = \inf \{ t>0 : W_t \geq a \}
\end{equation}
(We are really looking for $W_t=a$)

We can instead think
\begin{equation}
	\{ \tau_a\leq t \} = \{ \max_{0\leq s \leq t} W_s \geq a \}
\end{equation}
Then
\begin{equation}
	\tilde{W}_t = W_t \Theta_{t \leq \tau} + (2a - W_t) \Theta_{t > \tau}
\end{equation}
is a Brownian Motion. RP.

Now suppose we want to find the probability:
\begin{align}
	\mathbb{P}(\tau& \leq t, |W_t-a|> b) \\
	=& \mathbb{P}(\tau \leq t, W_t > a+b) + \mathbb{P}(\tau \leq t, W_t < a-b)\\
	=& 2\mathbb{P}(\tau \leq t, W_t > a+b)\\
	=& 2N_{(0,t)}(-a-b)
\end{align}
But we also know that
\begin{equation}
	\mathbb{P}(\tau \leq t) = 2N_{(0,t)}(-a)
\end{equation}

A few remarks:
\begin{enumerate}
	\item $\tau$ is predictable. We say that $\tau$ is \emph{announced} by a sequence $\tau_n$, the hitting time of $a-\frac1n$.
	\item Large Deviations: Consider the probability of a very large deviation ($a \to \infty$)
	\begin{equation}
		\mathbb{P}(\tau_a \leq t) \sim e^{-\frac{a^2}{2t}}
	\end{equation}
	where we mean this in term of the log.
\end{enumerate}


\chapter{Systemic Risk 2 - Fouque}
\section{Joint Distributions}
Consider the joint distribution of $(\tau,W_t)$. We can think of this as
\begin{equation}
	``\mathbb{P}(\tau\leq t, W_t = v)'' = \begin{cases}
		\mathbb{P}(W_t = v) & \text{if } v > a\\
		\mathbb{P}(W_t=2a-v) & \text{if } v < a
	\end{cases} = \begin{cases}
		g(v)\\
		g(2a-v)
	\end{cases}
\end{equation}

For nonstandard Brownian Motions,
\begin{equation}
	X_t = x + \mu t + \sigma W_t
\end{equation}
And then
\begin{equation}
	\tau = \inf \{ t> 0, x_t \geq a \}
\end{equation}
So then,
\begin{align}
	x_t \geq a  \iff& (x+\mu t + \sigma W_t \geq a)\\
	\iff& \mu t + \sigma W_t \geq a-x\\
	\iff& \frac{\mu}{\sigma} t + w_t \geq \frac{a-x}{\sigma}
\end{align}
Define $\theta=\frac{\mu}{\sigma}$, and $\tilde{a}=\frac{a-x}{\sigma}$.

Then we have that
\begin{equation}
	W_t + \theta t \equiv W_t^\theta
\end{equation}
Then Girsanov's Thm is 
\begin{equation}
	M_t^\theta = e^{-\theta W_t - \frac{1}{2}\theta^2t}
\end{equation}
Then
\begin{equation}
	\mathbb{E}(M_t^\theta) = 1
\end{equation}
and
\begin{equation}
	\frac{dP^\theta}{dP}|_{f_x} = M^\theta_t
\end{equation}

We can show that $W_t + \theta t \equiv W_t^\theta$ is a brownian motion under $\mathbb{P}^\theta$. So next,
\begin{equation}
	\mathbb{E}^\theta \left( x / F_s\right) = \frac{1}{M^\theta_s} \mathbb{E}(X M^\theta_t / F_s) 
\end{equation}
Then we will be able to show
\begin{equation}
	\mathbb{E}^\theta \left( e^{iu(W_t^\theta - W_s^\theta)} \right) = e^{-\frac{\mu^2}{2}(t-s)}
\end{equation}
Why do we care about all of that? Well, we can do the following:
\begin{equation}
	\tau = \inf \{ t> 0, W_t^\theta \geq \tilde{a} \}
\end{equation}
Then
\begin{equation}
	\mathbb{P}(\tau \leq t) = \mathbb{E}(\Theta_{\tau \leq t}) = \mathbb{E}^\theta \equiv \left( \Theta_{\tau \leq t} \frac{d\mathbb{P}}{d\mathbb{P}^\theta} \right)
\end{equation}
We can use the joint distribution of $(\tau, W_t^\theta)$.
\begin{equation}
	= N( -\frac{a - x - \mu t}{\sigma \sqrt{t}}) + e^{\frac{2(a-x)\mu)}{\sigma^2}} N(-\left( \frac{a - x + \mu t}{\sigma \sqrt{t}}\right))
\end{equation}

\section{Geometric Brownian Motion}
Now consider
\begin{equation}
	dS_t = S_t (\mu dt + \sigma dW_t)
\end{equation}
which is a stochastic differential equation. We should think of this as a convient notation for something more complicated. Basically,
\begin{equation}
	S_t dW_t \to \int_0^t S_a dW_a
\end{equation}

We will use Ito's lemma which lets us do the chain rule with a brownian motion:
\begin{equation}
	dg(W_t) = g'(W_t) dW_t + \frac12 g''(W_t)dt
\end{equation}
where $g(t,W_t)$ is once differentiable in $t$ and twice differentiable in $W_t$. Then if we have
\begin{equation}
	dX_t = \varphi_t dt + \psi_t dW_t
\end{equation}
then
\begin{equation}
	dg(X_t)  = g'(X_t)dX_t + \frac{1}{2} g''(X_t)d\langle x\rangle_t
\end{equation}
Then an SDE like
\begin{equation}
	dX_t = b(t,X_t)dt + \sigma(t,X_t) dW_t
\end{equation}
has a solution like
\begin{equation}
	S_t = S_0 e^{(\mu - \frac{\sigma^2}{2})t} + \sigma W_t
\end{equation}

\section{Passage Time}
Now consider
\begin{equation}
	dS_t = S_t(\mu dt + \sigma dW_t) = S_t(r dt + \sigma(dW_t + \frac{\mu-r}{\sigma}dt))
\end{equation}
well then, $\mathbb{P}^*$: $e^{-rt}S_t$ is a martingale.

\section{Defaultable Bond}
Now we want to have a defaultable bond. The bond starts at $S_0$ and has a default value of $D$(if it touches $D$ before maturity time $T$ then we have no payout, otherwise get 1.) So then the price of the bond is
\begin{equation}
	\mathbb{P}^D_{(0,T)} = \mathbb{E}^*(\Theta_{\tau > T}) = \mathbb{P}^*(\tau > T).
\end{equation}
So now we consider
\begin{equation}
	\{ \inf_{0 < t \leq T} S_0 e^{(r- \frac{\sigma^2}{2})t + \sigma W_t} > 0 \} 
\end{equation}
By taking the logarithm, we get a nonstandard brownian motion, which we can evaluate. By some computation, we can get
\begin{equation}
	\mathbb{P}^\theta(0,T) = e^{-rT}\left[ N(d_2^+) - \left(\frac{S_0}{D} \right)^{1-\frac{2r}{\sigma^2}}N(d_2^-) \right]
\end{equation}
with
\begin{equation}
	d_2^{\pm} = \frac{\pm \log \frac{S_0}{D} + \left(r - \frac{\sigma^2}{2} \right)t}{\sigma \sqrt{t}}
\end{equation}

The main ingredients here were reflection principle and change of measure. Alternatively, one could do this completely with partial differential equations.

\section{Yield}
We can represent the yield by
\begin{equation}
	y(0,t) = -\frac{1}{T}\log \frac{\mathbb{P}^D(0,T)}{\mathbb{P}(0,T)}
\end{equation}

\subsection{Example}
Consider some bond $B_1$ with a 10\% default rate. This is too risky for index funds, and not risky enough for hedge fund.

A financial engineer will do this: Buy two bonds, $B_1$ and $B_2$ (both with 10\% default rate), then stack them. Now
\begin{enumerate}
	\item pays if 2 defaults: 0.99
	\item pays if 1 default: .8
\end{enumerate}
What's the problem? We assumed independence.

In credit, the main risk was correlation of default. This is not really computable.

Consider two processes $W_t^{(1)}$, $W_t^{(2)}$ and the hitting times $\tau^{(1)}$ and $\tau^{(2)}$ try and figure the probability $\mathbb{P}(z^{(1)} > T; z^{(2)}> T)$, well, it's hard to quantify. If 

\section{Systemic Risk}
Usually we will consider the log-capitalization of Banks, and in particular multiple banks:
\begin{equation}
	dX_t^i = \mu_t^i dt + \sigma^i dW_t^i.
\end{equation}
\subsection{Toy Model}
We can consider the drift of capitalization, which includes some coupling of the systems.
\begin{equation}
	dX^i_t = \frac{a}{N} \sum_{j\ne i} (X^j_t - X_t^i)dt + \sigma^i (\rho dW_t^\theta + \sqrt{1-\rho^2} dW_t^i)
\end{equation}
where the $W^i_t$ are independent.
What is the intuition here? That borrowing and lending goes on between the banks. This is related to flocking and swarming models.

What other mathematical model behaves this way(where we have randomness but also attraction)? ornstein uhlenbeck process:
\begin{equation}
	dY_t = a(m-y_t)dt + \sigma dW_t,
\end{equation}
which we know how to solve:
\begin{equation}
	Y_t = m + (y-m)e^{-at} + \sigma e^{-at} \int_0^t e^{as}dW_p
\end{equation}
And we know that this is $\mathcal{N}(m, \frac{\sigma^2}{2a})$.

Next step is:
\begin{equation}
	d(\frac{1}{N} \sum_{i=1}^N X_t^i ) = 0 dt + \sigma \rho dW_t^0 + \frac{\sigma\sqrt{1-\rho^2}}{N} \sum_{i=1}^N dW_t^i.
\end{equation}
Now consider if $X_0^i=x_0^i=0$, which gives
\begin{equation}
	\bar{x}_t = \frac{1}{N}\sum_{i=1}^N X_t^i = \sigma \rho W_t^0 + \frac{\sigma \sqrt{1-\rho^2}}{N} \sum_{i=1}^N W_t^i
\end{equation}
If we take $\rho=0$ for a second, then
\begin{equation}
	\bar{x}_T \sim \frac{\sigma}{\sqrt{N}}\tilde{W}_t
\end{equation}